\chapter*{Chapitre 2 : Architecture et conception du système}
\addcontentsline{toc}{chapter}{Chapitre 2 : Architecture et conception du système}
\thispagestyle{fancy}
\setcounter{section}{0}
\newpage

\section{Approche architecturale globale}

La conception de nos deux systèmes complémentaires a été guidée par des principes modernes d'architecture logicielle, visant à créer des solutions robustes, évolutives et maintenables.

\subsection{Vision architecturale}

Notre approche architecturale repose sur plusieurs principes fondamentaux :

\begin{itemize}
  \item \textbf{Séparation des préoccupations} : Distinction claire entre les différentes couches et composants du système
  
  \item \textbf{Modularité} : Organisation en modules indépendants et faiblement couplés
  
  \item \textbf{Scalabilité horizontale} : Capacité à augmenter les performances en ajoutant des instances de service
  
  \item \textbf{Résilience} : Tolérance aux pannes et récupération automatique
  
  \item \textbf{Sécurité par conception} : Intégration des considérations de sécurité à chaque niveau
\end{itemize}

\subsection{Architecture du système de gestion scolaire}

Le système de gestion scolaire adopte une architecture multicouche moderne avec une séparation claire entre frontend et backend :

\begin{figure}[H]
  \centering
  \includegraphics[width=0.9\textwidth,keepaspectratio]{pfe-pics/diagrames/archetecture.png}
  \caption{\textbf{Architecture globale} du système de gestion scolaire.}
  \label{fig:school_architecture}
\end{figure}

Cette architecture se compose de :

\begin{itemize}
  \item \textbf{Couche présentation} : Applications web et mobile offrant des interfaces adaptées à chaque type d'utilisateur
  
  \item \textbf{Couche API} : Services RESTful sécurisés exposant les fonctionnalités du système
  
  \item \textbf{Couche métier} : Implémentation de la logique métier et des règles de gestion
  
  \item \textbf{Couche persistance} : Gestion des données et interactions avec la base de données
  
  \item \textbf{Services transversaux} : Authentification, journalisation, gestion des fichiers, etc.
\end{itemize}

\subsection{Architecture du système de création de profils IA}

Le système de création de profils IA s'appuie sur une architecture orientée services, optimisée pour le traitement de documents et l'interaction avec des modèles d'IA :

\begin{itemize}
  \item \textbf{Frontend React/TypeScript} : Interface utilisateur réactive pour la gestion des profils et l'interaction avec l'IA
  
  \item \textbf{Backend FastAPI} : Services Python haute performance pour le traitement des documents et l'orchestration des modèles d'IA
  
  \item \textbf{Pipeline de traitement} : Système de traitement asynchrone pour l'extraction et l'analyse des contenus documentaires
  
  \item \textbf{Stockage hybride} : Combinaison de base de données PostgreSQL pour les métadonnées et de stockage objet pour les documents
  
  \item \textbf{Intégration IA} : Connecteurs vers des services d'IA externes (OpenRouter) pour le traitement du langage naturel
\end{itemize}

\section{Conception détaillée du système de gestion scolaire}

\subsection{Modèle de données}

Le modèle de données du système de gestion scolaire a été conçu pour représenter efficacement toutes les entités du domaine éducatif et leurs relations :

\begin{figure}[H]
  \centering
  \includegraphics[width=0.9\textwidth,keepaspectratio]{pfe-pics/diagrames/tabaales.png}
  \caption{\textbf{Modèle de données} du système de gestion scolaire.}
  \label{fig:school_data_model}
\end{figure}

Les principales entités du modèle sont :

\begin{itemize}
  \item \textbf{User} : Entité de base pour tous les utilisateurs du système, avec spécialisation par rôle
  
  \item \textbf{Student} : Informations spécifiques aux étudiants, incluant leur parcours académique
  
  \item \textbf{Teacher} : Données relatives aux enseignants, incluant leurs spécialités et disponibilités
  
  \item \textbf{Parent} : Informations sur les parents et leurs relations avec les étudiants
  
  \item \textbf{Course} : Définition des cours avec leur contenu et organisation
  
  \item \textbf{Attendance} : Enregistrement des présences aux séances de cours
  
  \item \textbf{Assignment} : Travaux et évaluations assignés aux étudiants
  
  \item \textbf{Grade} : Notes et évaluations des étudiants
  
  \item \textbf{Notification} : Messages et alertes envoyés aux utilisateurs
  
  \item \textbf{Document} : Ressources pédagogiques et administratives partagées
\end{itemize}

\subsection{Diagramme de classes}

Le diagramme de classes détaille la structure objet du système et les relations entre les différentes classes :

\begin{figure}[H]
  \centering
  \includegraphics[width=0.9\textwidth,keepaspectratio]{pfe-pics/diagrames/class.png}
  \caption{\textbf{Diagramme de classes} du système de gestion scolaire.}
  \label{fig:school_class_diagram}
\end{figure}

Ce diagramme met en évidence :

\begin{itemize}
  \item Les attributs et méthodes clés de chaque classe
  
  \item Les relations d'héritage, notamment pour les différents types d'utilisateurs
  
  \item Les associations entre classes, avec leur cardinalité
  
  \item Les agrégations et compositions représentant les relations de contenance
\end{itemize}

\subsection{Architecture frontend}

L'architecture frontend du système de gestion scolaire repose sur une approche modulaire et réactive :

\subsubsection{Application web}

L'application web utilise React avec TypeScript et adopte une architecture basée sur les composants :

\begin{itemize}
  \item \textbf{Structure des composants} : Organisation hiérarchique avec composants atomiques, moléculaires et organismes
  
  \item \textbf{Gestion d'état} : Utilisation de Context API et hooks personnalisés pour la gestion de l'état global
  
  \item \textbf{Routage} : Système de navigation basé sur React Router avec gestion des autorisations
  
  \item \textbf{Thème et styles} : Approche basée sur Tailwind CSS avec thèmes personnalisables
  
  \item \textbf{Internationalisation} : Support multilingue avec gestion des traductions
\end{itemize}

\subsubsection{Application mobile}

L'application mobile est développée avec React Native, partageant une partie de la logique avec l'application web :

\begin{itemize}
  \item \textbf{Navigation native} : Utilisation de React Navigation pour une expérience fluide
  
  \item \textbf{Composants adaptatifs} : Interface optimisée pour les interactions tactiles
  
  \item \textbf{Fonctionnalités natives} : Intégration des notifications push, de l'appareil photo et du stockage local
  
  \item \textbf{Mode hors ligne} : Synchronisation des données pour utilisation sans connexion permanente
\end{itemize}

\subsection{Architecture backend}

Le backend du système de gestion scolaire est conçu selon une architecture en couches avec une séparation claire des responsabilités :

\begin{itemize}
  \item \textbf{Couche API} : Contrôleurs REST exposant les endpoints du système
  
  \item \textbf{Couche service} : Implémentation de la logique métier et orchestration des opérations
  
  \item \textbf{Couche repository} : Abstraction des opérations de persistance et requêtes à la base de données
  
  \item \textbf{Couche entité} : Modèles de données et mappings ORM
  
  \item \textbf{Services transversaux} : Authentification, autorisation, validation, journalisation
\end{itemize}

\subsection{Flux de travail principaux}

Plusieurs flux de travail clés ont été modélisés pour guider l'implémentation :

\subsubsection{Processus d'authentification}

\begin{figure}[H]
  \centering
  \includegraphics[width=0.9\textwidth,keepaspectratio]{pfe-pics/diagrames/State Diagram (for Authentication Flow).png}
  \caption{\textbf{Diagramme d'états} pour le processus d'authentification.}
  \label{fig:auth_flow}
\end{figure}

Ce diagramme illustre :

\begin{itemize}
  \item Les différents états possibles du processus d'authentification
  
  \item Les transitions entre ces états en fonction des actions utilisateur
  
  \item Les vérifications de sécurité à chaque étape
  
  \item La gestion des erreurs et des cas particuliers
\end{itemize}

\subsubsection{Gestion des cours}

\begin{figure}[H]
  \centering
  \includegraphics[width=0.7\textwidth,height=0.8\textheight,keepaspectratio]{pfe-pics/diagrames/Activity Diagram (for Course Management).png}
  \caption{\textbf{Diagramme d'activité} pour la gestion des cours.}
  \label{fig:course_management}
\end{figure}

Ce flux détaille :

\begin{itemize}
  \item Les étapes de création et configuration d'un cours
  
  \item L'assignation des enseignants et l'inscription des étudiants
  
  \item La gestion du matériel pédagogique
  
  \item Le suivi et l'évaluation des activités du cours
\end{itemize}

\subsubsection{Suivi des présences}

\begin{figure}[H]
  \centering
  \includegraphics[width=0.9\textwidth,keepaspectratio]{pfe-pics/diagrames/Attendance Tracking.png}
  \caption{\textbf{Diagramme de séquence} pour le suivi des présences.}
  \label{fig:attendance_tracking}
\end{figure}

Ce diagramme montre :

\begin{itemize}
  \item Les interactions entre l'enseignant et le système pour l'enregistrement des présences
  
  \item La validation et la persistance des données d'assiduité
  
  \item Les notifications automatiques en cas d'absence
  
  \item La consultation des rapports de présence par les différents acteurs
\end{itemize}

\subsubsection{Soumission et notation des devoirs}

\begin{figure}[H]
  \centering
  \includegraphics[width=0.9\textwidth,keepaspectratio]{pfe-pics/diagrames/Assignment Submission and Grading.png}
  \caption{\textbf{Diagramme de flux} pour la soumission et notation des devoirs.}
  \label{fig:assignment_grading}
\end{figure}

Ce flux illustre :

\begin{itemize}
  \item La création et l'assignation de travaux par l'enseignant
  
  \item La soumission des travaux par les étudiants
  
  \item Le processus d'évaluation et de feedback
  
  \item La publication et consultation des résultats
\end{itemize}

\subsubsection{Vérification parent-enfant}

\begin{figure}[H]
  \centering
  \includegraphics[width=0.9\textwidth,keepaspectratio]{pfe-pics/diagrames/Parent-Child Verification Process.png}
  \caption{\textbf{Diagramme de processus} pour la vérification parent-enfant.}
  \label{fig:parent_child_verification}
\end{figure}

Ce processus sécurisé détaille :

\begin{itemize}
  \item L'initiation de la demande d'association par le parent
  
  \item La vérification des informations fournies
  
  \item La validation par l'administration
  
  \item L'établissement de la relation parent-enfant dans le système
\end{itemize}

\subsubsection{Inscription aux cours}

\begin{figure}[H]
  \centering
  \includegraphics[width=0.9\textwidth,keepaspectratio]{pfe-pics/diagrames/Course Enrollment Process.png}
  \caption{\textbf{Diagramme de flux} pour l'inscription aux cours.}
  \label{fig:course_enrollment}
\end{figure}

Ce diagramme présente :

\begin{itemize}
  \item La recherche et sélection de cours par l'étudiant
  
  \item La vérification des prérequis et disponibilités
  
  \item Le processus d'approbation si nécessaire
  
  \item La confirmation de l'inscription et l'accès au matériel du cours
\end{itemize}

\subsection{Cas d'utilisation}

Le diagramme des cas d'utilisation offre une vue d'ensemble des fonctionnalités du système par type d'utilisateur :

\begin{figure}[H]
  \centering
  \includegraphics[width=0.9\textwidth,keepaspectratio]{pfe-pics/diagrames/usecase.png}
  \caption{\textbf{Diagramme de cas d'utilisation} du système de gestion scolaire.}
  \label{fig:use_cases}
\end{figure}

Ce diagramme met en évidence :

\begin{itemize}
  \item Les acteurs principaux du système (administrateur, enseignant, étudiant, parent)
  
  \item Les fonctionnalités accessibles à chaque type d'utilisateur
  
  \item Les relations entre les différents cas d'utilisation
  
  \item Les extensions et inclusions entre cas d'utilisation
\end{itemize}

\section{Conception détaillée du système de création de profils IA}

\subsection{Architecture des composants}

L'architecture du système de création de profils IA s'articule autour de plusieurs composants spécialisés :

\begin{figure}[H]
  \centering
  \includegraphics[width=0.9\textwidth,keepaspectratio]{pfe-pics/diagrames/Component Diagram (showing the system_s architecture).png}
  \caption{\textbf{Diagramme de composants} du système de création de profils IA.}
  \label{fig:ai_components}
\end{figure}

Les principaux composants sont :

\begin{itemize}
  \item \textbf{Gestionnaire de profils} : Création et configuration des profils IA
  
  \item \textbf{Processeur de documents} : Traitement et extraction d'informations à partir des documents uploadés
  
  \item \textbf{Moteur d'indexation} : Organisation des connaissances pour une recherche efficace
  
  \item \textbf{Interface conversationnelle} : Gestion des interactions avec les profils IA
  
  \item \textbf{Gestionnaire d'API} : Exposition des fonctionnalités via une API RESTful
  
  \item \textbf{Système d'authentification} : Gestion des utilisateurs et des accès
\end{itemize}

\subsection{Pipeline de traitement des documents}

Le pipeline de traitement des documents constitue un élément central du système de création de profils IA :

\begin{figure}[H]
  \centering
  \includegraphics[width=0.6\textwidth,keepaspectratio]{pfe-pics/diagrames/Pipeline de Traitement des Documents (Document Processing Pipeline).png}
  \caption{\textbf{Diagramme de flux} du pipeline de traitement des documents.}
  \label{fig:document_processing_pipeline}
\end{figure}

Ce pipeline se compose des étapes suivantes :

\begin{itemize}
  \item \textbf{Étape 1 - Extraction de texte} : Conversion des différents formats de documents en texte brut
  
  \item \textbf{Étape 2 - Analyse structurelle} : Identification des sections, titres, paragraphes et éléments spéciaux
  
  \item \textbf{Étape 3 - Enrichissement sémantique} : Détection des entités, concepts et relations
  
  \item \textbf{Étape 4 - Chunking intelligent} : Segmentation du contenu en unités de connaissance optimales
  
  \item \textbf{Étape 5 - Indexation vectorielle} : Création d'embeddings pour la recherche sémantique
  
  \item \textbf{Étape 6 - Validation et stockage} : Vérification de la qualité et persistance des données traitées
\end{itemize}

\subsection{Modèle de données}

Le modèle de données du système de création de profils IA est conçu pour gérer efficacement les profils, documents et interactions :

\begin{figure}[H]
  \centering
  \includegraphics[width=0.9\textwidth,keepaspectratio]{pfe-pics/diagrames/Modèle de Données du Système de Profils IA (AI Profiles Data Model).png}
  \caption{\textbf{Modèle de données} du système de création de profils IA.}
  \label{fig:ai_data_model}
\end{figure}

Les principales entités du modèle sont :

\begin{itemize}
  \item \textbf{Profile} : Définition d'un profil IA avec ses paramètres et métadonnées
  
  \item \textbf{Document} : Information sur les documents sources avec leur statut de traitement
  
  \item \textbf{Chunk} : Segments de contenu extraits des documents
  
  \item \textbf{Embedding} : Représentations vectorielles des chunks pour la recherche sémantique
  
  \item \textbf{Conversation} : Sessions de discussion avec un profil IA
  
  \item \textbf{Message} : Échanges individuels au sein d'une conversation
  
  \item \textbf{ApiKey} : Clés d'accès pour l'intégration externe
  
  \item \textbf{User} : Utilisateurs du système avec leurs permissions
\end{itemize}

\subsection{Flux de traitement des requêtes}

Le traitement d'une requête adressée à un profil IA suit un flux optimisé :

\begin{figure}[H]
  \centering
  \includegraphics[width=0.9\textwidth,keepaspectratio]{pfe-pics/diagrames/Flux de Conversation IA (AI Conversation Flow).png}
  \caption{\textbf{Diagramme de séquence} montrant le flux de conversation avec un profil IA.}
  \label{fig:ai_conversation_flow}
\end{figure}

Ce flux se décompose en plusieurs étapes clés :

\begin{itemize}
  \item \textbf{Réception de la requête} : Validation et prétraitement de la question utilisateur
  
  \item \textbf{Recherche sémantique} : Identification des chunks les plus pertinents dans la base de connaissances
  
  \item \textbf{Construction du contexte} : Assemblage des informations pertinentes avec l'historique de conversation
  
  \item \textbf{Génération de réponse} : Utilisation du modèle d'IA avec le contexte pour produire une réponse
  
  \item \textbf{Post-traitement} : Formatage, vérification et enrichissement de la réponse
  
  \item \textbf{Enregistrement} : Sauvegarde de l'échange pour référence future et amélioration
\end{itemize}

\section{Architecture d'intégration des deux systèmes}

\subsection{Points d'intégration conceptuels}

L'intégration entre le système de gestion scolaire et le système de création de profils IA s'effectue à plusieurs niveaux architecturaux :

\begin{itemize}
  \item \textbf{Authentification unifiée} : Système SSO permettant une navigation fluide entre les deux plateformes
  
  \item \textbf{Association profils-cours} : Possibilité d'associer des profils IA spécifiques à des cours
  
  \item \textbf{Partage de ressources} : Utilisation des documents pédagogiques du système scolaire comme source pour les profils IA
  
  \item \textbf{Intégration UI} : Widgets permettant d'interagir avec les profils IA directement depuis l'interface du système scolaire
\end{itemize}

\subsection{Diagrammes de dépendances}

Les diagrammes de dépendances suivants illustrent les relations entre les différents composants de chaque système :

\begin{figure}[H]
  \centering
  \includegraphics[width=1.0\textwidth,keepaspectratio]{pfe-pics/diagrames/project_1_dep_diagrame.png}
  \caption{\textbf{Diagramme de dépendances du système de gestion scolaire} montrant les composants et leurs relations.}
  \label{fig:school_dependencies}
\end{figure}

\begin{figure}[H]
  \centering
  \includegraphics[width=1.0\textwidth,keepaspectratio]{pfe-pics/diagrames/project_ai_profile_cretion_dep_diagrame.png}
  \caption{\textbf{Diagramme de dépendances du système de création de profils IA} montrant les composants et leurs relations.}
  \label{fig:ai_dependencies}
\end{figure}

\subsection{Architecture de communication}

La communication entre ces deux systèmes complémentaires s'effectue à travers plusieurs mécanismes architecturaux :

\begin{itemize}
  \item \textbf{API Gateway} : Point d'entrée unifié gérant le routage vers les services appropriés
  
  \item \textbf{Services d'identité partagés} : Gestion centralisée des utilisateurs et autorisations
  
  \item \textbf{Bus d'événements} : Communication asynchrone entre les systèmes pour les mises à jour et notifications
  
  \item \textbf{Cache distribué} : Optimisation des performances pour les données fréquemment accédées
\end{itemize}

\section{Considérations architecturales transversales}

\subsection{Stratégie de sécurité}

La sécurité a été intégrée à tous les niveaux de la conception architecturale :

\begin{itemize}
  \item \textbf{Authentification robuste} : Mécanismes d'authentification modernes avec support MFA
  
  \item \textbf{Autorisation fine} : Contrôle d'accès basé sur les rôles et les ressources
  
  \item \textbf{Protection des données} : Chiffrement des données sensibles au repos et en transit
  
  \item \textbf{Validation des entrées} : Filtrage et validation de toutes les entrées utilisateur
  
  \item \textbf{Audit et journalisation} : Traçabilité des actions sensibles pour détection d'anomalies
  
  \item \textbf{Protection contre les attaques courantes} : Mesures contre XSS, CSRF, injection SQL, etc.
\end{itemize}

\subsection{Approche de performance}

L'architecture prend en compte les considérations de performance à travers plusieurs stratégies :

\begin{itemize}
  \item \textbf{Mise en cache multicouche} : Cache au niveau du navigateur, de l'API et de la base de données
  
  \item \textbf{Traitement asynchrone} : Utilisation de files d'attente pour les opérations longues
  
  \item \textbf{Pagination et chargement différé} : Optimisation du chargement des données volumineuses
  
  \item \textbf{Optimisation des requêtes} : Indexation et requêtes efficientes sur la base de données
  
  \item \textbf{Distribution de charge} : Répartition du trafic entre plusieurs instances de service
\end{itemize}

\section{Conclusion de la conception}

La conception détaillée présentée dans ce chapitre établit une base solide pour l'implémentation de nos deux systèmes complémentaires. L'approche architecturale adoptée privilégie :

\begin{itemize}
  \item La modularité et l'extensibilité pour faciliter l'évolution future
  
  \item La séparation claire des responsabilités pour une maintenance simplifiée
  
  \item L'intégration harmonieuse des deux systèmes pour une expérience utilisateur cohérente
  
  \item La sécurité et la performance comme préoccupations transversales
\end{itemize}

Les diagrammes et modèles élaborés serviront de guide pour la phase d'implémentation, tout en laissant la flexibilité nécessaire pour s'adapter aux défis techniques qui pourraient survenir durant le développement.

L'architecture proposée répond aux exigences fonctionnelles et non fonctionnelles identifiées dans le cahier des charges, et pose les fondations pour des systèmes robustes, évolutifs et centrés sur l'utilisateur. 
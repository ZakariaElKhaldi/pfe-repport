% Timeline diagrams for PFE Project
% Including PERT diagram for project planning

\documentclass[12pt, a4paper]{article}

% --- Packages ---
\usepackage[utf8]{inputenc}
\usepackage[T1]{fontenc}
\usepackage[french]{babel}
\usepackage{tikz}
\usetikzlibrary{positioning,arrows.meta,shapes.geometric}
\usepackage{xcolor}
\usepackage{geometry}
\geometry{a4paper, margin=2cm}

\begin{document}

\section*{Planification du projet}

% Note that we're using the Mermaid chart instead of the LaTeX Gantt chart
% because it's easier to handle and doesn't have formatting issues

\subsection*{Diagramme PERT}

Le diagramme PERT ci-dessous illustre les dépendances entre les différentes étapes du projet et le chemin critique.

\vspace{1cm}
\begin{center}
\begin{tikzpicture}[
  node distance=2.5cm,
  thick,
  event/.style={circle, draw, minimum size=1.2cm},
  critical/.style={circle, draw, fill=red!20, minimum size=1.2cm},
  activity/.style={-Stealth, thick},
  critical path/.style={-Stealth, thick, draw=red}
]

% Nœuds (événements)
\node[critical] (1) {1};
\node[critical] (2) [right=of 1] {2};
\node[critical] (3) [right=of 2] {3};
\node[critical] (4) [right=of 3] {4};
\node[critical] (5) [below right=of 4] {5};
\node[event] (6) [above right=of 4] {6};
\node[critical] (7) [right=of 5] {7};

% Activités (flèches)
\draw[critical path] (1) -- node[midway, above] {2 sem.} node[midway, below] {Choix du sujet} (2);
\draw[critical path] (2) -- node[midway, above] {3 sem.} node[midway, below] {Cahier des charges} (3);
\draw[critical path] (3) -- node[midway, above] {3 sem.} node[midway, below] {Conception} (4);
\draw[critical path] (4) -- node[midway, above] {2 sem.} node[midway, below] {Choix des techs} (5);
\draw[activity] (4) -- node[midway, above] {Parallèle} node[midway, below] {Doc. technique} (6);
\draw[critical path] (5) -- node[midway, above] {10 sem.} node[midway, below] {Dév. Phase 1} (7);
\draw[activity] (6) to [bend right] node[midway, above] {Support} (7);

% Indiquer l'arrêt du développement
\node[draw=red, dashed, circle, minimum size=1.5cm, right=of 7] (8) {Arrêt};
\draw[activity, red, dashed] (7) -- node[midway, above] {Examens} (8);

\end{tikzpicture}
\end{center}

\vspace{1cm}
\textbf{Légende:}
\begin{itemize}
  \item \textbf{Nœuds rouges:} Événements sur le chemin critique
  \item \textbf{Flèches rouges:} Activités sur le chemin critique
  \item Le développement Phase 2 et les tests n'ont pas été complétés en raison de l'interruption du projet pour les examens nationaux
\end{itemize}

\vspace{1cm}
\textbf{Durées ajustées des étapes:}

\begin{center}
\begin{tabular}{|l|l|c|c|}
\hline
\textbf{Étape} & \textbf{Description} & \textbf{Dates} & \textbf{Durée} \\
\hline
1. Choix du sujet & Sélection du sujet et validation & 03/11/2024 - 17/11/2024 & 2 semaines \\
\hline
2. Cahier des charges & Rédaction des besoins et contraintes & 18/11/2024 - 08/12/2024 & 3 semaines \\
\hline
3. Conception & Maquettes et architecture & 09/12/2024 - 29/12/2024 & 3 semaines \\
\hline
4. Choix technologies & Sélection des langages et frameworks & 30/12/2024 - 12/01/2025 & 2 semaines \\
\hline
5. Développement (P1) & Fonctionnalités de base & 13/01/2025 - 24/03/2025 & 10 semaines \\
\hline
6. Développement (P2) & \begin{tabular}{@{}c@{}}Intégration finale\\ (partiellement réalisée)\end{tabular} & 25/03/2025 - 01/04/2025 & \begin{tabular}{@{}c@{}}1 semaine\\ (sur 6 prévues)\end{tabular} \\
\hline
7. Tests & \begin{tabular}{@{}c@{}}Tests d'intégration\\ (non réalisés)\end{tabular} & - & - \\
\hline
\end{tabular}
\end{center}

\end{document} 